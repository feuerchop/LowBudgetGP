% This is LLNCS.DEM the demonstration file of
% the LaTeX macro package from Springer-Verlag
% for Lecture Notes in Computer Science,
% version 2.4 for LaTeX2e as of 16. April 2010
%
\documentclass{llncs}
%
\usepackage{makeidx}  % allows for indexgeneration
% For figures
\usepackage{graphicx} % more modern
%\usepackage{epsfig} % less modern
\usepackage{subfigure} 
%\usepackage{subfig}%[caption=false,font=footnotesize]

% For citations
\usepackage{natbib}

% For algorithms
\usepackage{algorithm}
\usepackage{algorithmic}

% As of 2011, we use the hyperref package to produce hyperlinks in the
% resulting PDF.  If this breaks your system, please commend out the
% following usepackage line and replace \usepackage{icml2012} with
% \usepackage[nohyperref]{icml2012} above.
\usepackage{hyperref}

% Packages hyperref and algorithmic misbehave sometimes.  We can fix
% this with the following command.
\newcommand{\theHalgorithm}{\arabic{algorithm}}

%-------------------------------------------------------------
%                      Own Commands
%-------------------------------------------------------------
\usepackage{amsmath}
\newcommand{\cbn}{\textsc{Cbn}}
\newcommand{\bn}{\textsc{Bn}}

\renewcommand{\algorithmiccomment}[1]{/* #1 */}
\def\ci{\perp\!\!\!\perp}
\def\dep{\perp\!\!\!\perp\!\!\!\!\!\!\!/\,\,\,\,}
% Theorem & Co environments and counters
%\newtheorem{theorem}{Theorem}[section]
%\newtheorem{lemma}[theorem]{Lemma}
%\newtheorem{corollary}[theorem]{Corollary}
%\newtheorem{remark}[theorem]{Remark}
%\newtheorem{definition}[theorem]{Definition}
\newtheorem{equat}[theorem]{Equation}
%\newtheorem{example}[theorem]{Example}
%-------------------------------------------------------------

\begin{document}
%
\frontmatter          % for the preliminaries
%
%\pagestyle{headings}  % switches on printing of running heads
%\addtocmark{Hamiltonian Mechanics} % additional mark in the TOC
%%
%\tableofcontents
%
\mainmatter              % start of the contributions
%
\title{Communication Efficient Learning towards Crowdsourcing Applications}
%
%\titlerunning{Hamiltonian Mechanics}  % abbreviated title (for running head)
%                                     also used for the TOC unless
%                                     \toctitle is used
%
\author{Bojan  Kolosnjaji\inst{1} \and Huang Xiao\inst{2} 
	\and Claudia Eckert\inst{1} }
%Jeffrey Dean \and David Grove \and Craig Chambers \and Kim~B.~Bruce \and
%Elsa Bertino}
%
\authorrunning{Bojan Kolosnjaji et al.} % abbreviated author list (for running head)
%
%%%% list of authors for the TOC (use if author list has to be modified)
%\tocauthor{Ivar Ekeland, Roger Temam, Jeffrey Dean, David Grove,
%Craig Chambers, Kim B. Bruce, and Elisa Bertino}
%
\institute{The great instititue, \\
	Some where on earth, \\
\email{yourname@email.com}
%\\ WWW home page:
%\texttt{https://www.aisec.fraunhofer.de/}
%\and
%Universit\'{e} de Paris-Sud,
%Laboratoire d'Analyse Num\'{e}rique, B\^{a}timent 425,\\
%F-91405 Orsay Cedex, France
}

\maketitle              % typeset the title of the contribution

\begin{abstract}
submit  abstract until 13.04

\keywords{key, word, add, here, ...}
\end{abstract}
%

% ---- Bibliography ----
%
\bibliographystyle{plain}
\bibliography{literature} % use this
%\clearpage
%\addtocmark[2]{Author Index} % additional numbered TOC entry
%\renewcommand{\indexname}{Author Index}
%\printindex
%\clearpage
%\addtocmark[2]{Subject Index} % additional numbered TOC entry
%\markboth{Subject Index}{Subject Index}
%\renewcommand{\indexname}{Subject Index}
%\input{subjidx.ind}
\end{document}
